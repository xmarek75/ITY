\documentclass[a4paper, 11pt]{article}
\usepackage[utf8]{inputenc}
\usepackage[czech]{babel}
\usepackage{times}
\usepackage[left=2cm, top=3cm, text={17cm,24cm}]{geometry}
\usepackage{color}
\usepackage[unicode]{hyperref}
\usepackage{url}
\DeclareUrlCommand\url{\def\UrlLeft{<}\def\UrlRight{>} \urlstyle{tt}}
\title{proj4}
\author{xmarek75}
\date{April 2021}

\begin{document}



\begin{titlepage}

\begin{center}
\Huge
\textsc{Vysoké učení technické v~Brně\\
\huge
Fakulta informačních technologií} \\
\vspace{\stretch{0.25}}

\LARGE
Typografie a publikování\,--\,4.~projekt \\
\Huge
Bibliografické citace
\vspace{\stretch{0.4}}
\end{center}

{\Large \today \hfill
Pavel Marek }
\end{titlepage}
\section*{Předmluva}
Tento dokument byl vytvořen za účelem splnění projektu do předmětu ITY.\\
Projekt se týká Bibliografických citací.
\subsection*{Pojem \emph{Bibliografické citace}}
Jedná se o~souhrn údajů vztahujících se k~citované publikaci nebo části této publikace. Bibliografická citace slouží k~identifikaci citované publikace.
Autor každé odborné publikace musí dle autorského zákona uvést zdroje, ze kterých čerpal podklady pro obsah své publikace. Více o bibliografických citací v \cite{Pysny}
\section{Úvod do systému {\LaTeX}}
\subsection{Definice}
{\LaTeX} je typografickým systémem, který je určen k sazbě vědeckých a~matematických dokumentů vysoké typografické kvality. Systém je rovněž vhodný pro tvorbu všech možných druhů jiných dokumentů, od jednoduchých dopisů po složité knihy. Systém {\LaTeX} je postaven na typografickém formátovacím programu {\TeX} Donalda~E.~Knutha. Více o~historii a~filosofii v \cite{Oetiker}
\subsection{Proč je dobré používat \LaTeX}
V článku \cite{Simecek}, vyzdvihuje autor jako hlavní důvod proč používat {\LaTeX} jeho "neomezenost", v {\LaTeX}u máme možnost nadefinovat si přesně to, co chceme.
Velkou výhodou je i to, že \LaTeX je zdarma na rozdíl od některých komerčních produktů, za něž se musí platit vysoké částky.\\
V časopise EdTech KISK \cite{Davidek} zase považuje autor za největší výhodu jeho výstupy (tedy pdf soubory připravené pro tisk), že jsou po estetické, tak i~po typografické stránce na výrazněji profesionálnější úrovni než například MS Word. Jde například o kontrasty a velikosti mezer napříč jednotlivými nadpisy, zalomení řádků, odstavců, stránek atd.  
\subsection{Práce v {\LaTeX}u}
Tvorba dokumentů pomocí {\LaTeX}u připomíná programovaní podobné HTML.\\
Podle \cite{Rybicka} je složeno ze tří hlavních částí:
\begin{itemize}
    \item psaní a úprava zdrojového textu
    \item překlad\,-\,vysázení 
    \item prohlížení
\end{itemize}
\subsection{Struktura dokumentu}
{\large Základní formát dokumentu:}\\
\verb|\documentclass[volby]{styl}|\\
 \dots \textcolor{blue}{preambule}\\
 \verb|\begin{document}|\\
 \dots \textcolor{blue}{textová část}\\
 \verb|\end{document}|
 \begin{itemize}
     \item \textbf{Styl} – Definuje styl sazby, jakým má být dokument vypracován.
     \item \textbf{Volby} – Volitelné parametry,umožňují modifikovat vlastnosti stylu.
     \item \textbf{Preambule} – V~preambuli, se nacházejí příkazy, jejichž platnost se vztahuje k~celému textu.
     \item \textbf{Textová část} – Textová část dokumentu obsahuje text a~příkazy, které ovlivňují formátování a~způsob zobrazení textu.
 \end{itemize}
 Více o tom, co obsahují jednotlivé části v \cite{Benes}
 
\subsection{Příkazy}
Klíčovou roli mají příkazy, jimiž řídíte celou sazbu. Příkaz vždy začíná
zpětným lomítkem. Podle toho, co následuje, se dělí do dvou kategorií:
\begin{itemize}
    \item \textbf{Řídící slova} jsou tvořena libovolně dlouhou posloupností písmen
    Jako příklad může posloužit příkaz \verb|\TeX|, který
vysází logo \TeX. Řídicí slovo končí prvním nepísmenným znakem.
    \item \textbf{Řídící znaky} obsahují jediný nepísmenný znak. Například \verb|\#|, kterým se sází znak „\#“, je řídicí znak.
\end{itemize}
Více o příkazech v \cite{Satrapa}
\subsection{Matematické formuly}
V knize \emph{A Guide to \LaTeX{} and Electronic Publishing}\cite{Kopka} popisují matematiku jako "duši" systému \TeX. Protože sazba matematických zápisů je v normálním tisku velmi komplikovaná, ještě více komplikovaná byla na psacím stroji.
A~proto Donald Knuth vytvořil svůj vlastní formátovací jazyk.
\subsubsection{Matematické prostředí}
Pro sazbu matematiky používá {\TeX} zvláštní mód\,–\,matematický. Tento mód
se dělí na matematický textový a matematický vysazený. Více v \cite{Wyrwolova}
\subsection{Vkládání obrázků}
Pro práci s obrázky je vhodné vyvolat balíček \texttt{graphicx}, tzn. v~preambuli dokumentu napsat\\*  {\verb|\usepackage|\verb|{graphicx}|}. U~tohoto balíčku můžeme navolit parametry typu \texttt{pdftex}, \texttt{dvips}, aj., což jsou ovladače pro překladače {\LaTeX} a PDF{\LaTeX} (a~další), a~které umožňují  vytvářet dokumenty ve~zvoleném formátu. Více o překladačích v \cite{Polasek}.\\*
Při vkládání obrázku je dobré zvolit vhodné parametry:
\begin{itemize}
    \item \textbf{angle}\,-\,natočení obrázku o libovolný počet stupňů, a to v kladném i záporném směru
    \item \textbf{height}\,-\,nastavení výšky obrázku
    \item \textbf{width}\,-\,nastavení šířky obrázku
    \item \textbf{scale}\,-\,změna velikosti obrázku poměrně ke stranám, výsledná velikost bude násobkem zadané hodnoty a původní velikosti
    \item \textbf{type}\,-\,definice typu vstupního souboru, nebude brán zřetel na příponu souboru s obrázkem
\end{itemize}
Jednotlivé parametry oddělujeme mezi sebou čárkou a hodnoty přiřazujeme rovnítkem, například.:\\
\verb|angle=-45, height=7cm, width=35mm|\\
Podrobnější popis parametrů v \cite{Svamberg}.



    
\newpage
    \bibliographystyle{czechiso}
	\renewcommand{\refname}{Literatura}
	\bibliography{proj4}

\end{document}