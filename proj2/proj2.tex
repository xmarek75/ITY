\documentclass[a4paper, 11pt, twocolumn]{article}

%   Pavel Marek <xmarek75@stud.fit.vutbr.cz>

\usepackage[utf8]{inputenc}
\usepackage[czech]{babel}
\usepackage[IL2]{fontenc}
\usepackage{changepage}
\usepackage{times}
\usepackage[unicode]{hyperref}
\usepackage{amsthm, amssymb, amsmath}
\usepackage[left=1.5cm, top=2.5cm, text={18cm, 25cm}]{geometry}
\usepackage{textcomp}

\theoremstyle{definition}
\newtheorem{definition}{Definice}

\newtheorem{veta}{Věta}




\begin{document}

\begin{titlepage}

\begin{center}
\Huge
\textsc{Fakulta informačních technologií\\
Vysoké učení technické v Brně} \\
\vspace{\stretch{0.25}}

\LARGE
Typografie a publikování -- 2. projekt \\
Sazba dokumentů a matematických výrazů \\
\vspace{\stretch{0.4}}
\end{center}

{\Large \the\year \hfill
Pavel Marek (xmarek75)}
\end{titlepage}

\section*{Úvod}

V~této úloze si vyzkoušíme sazbu titulní strany, matematic\-kých vzorců, prostředí a dalších textových struktur obvyk\-lých pro technicky zaměřené texty
(například rovnice~(\ref{rov1}) nebo Definice \ref{Definice} na straně \pageref{Definice}). Rovněž si vyzkoušíme pou\-žívání odkazů \verb|\ref| a \verb|\pageref|.

Na titulní straně je využito sázení nadpisu podle op\-tického středu s~využitím zlatého řezu. Tento postup byl probírán na přednášce. Dále je použito odřádkování se zadanou relativní velikostí 0.4 em a 0.3 em.

V případě, že budete potřebovat vyjádřit matematickou konstrukci nebo symbol a nebude se Vám dařit jej nalézt v~samotném {\LaTeX}u, doporučuji prostudovat možnosti ba\-líku maker \AmS-\LaTeX.

\section {Matematický text}

Nejprve se podíváme na sázení matematických symbolů a~výrazů v~plynulém textu včetně sazby definic a vět s~vy\-užitím balíku \verb|amsthm|. Rovněž použijeme poznámku pod čarou s použitím příkazu \verb|\footnote|. Někdy je vhodné použít konstrukci \verb|\mbox{}|, která říká, že text nemá být zalomen.

\begin{definition}
    \label{Definice} Rozšířený zásobníkový automat \emph{(RZA) je de\-finován jako sedmice tvaru A $= (Q$, $\Sigma$, $\Gamma$, $\delta$, $ q_0$, $Z_0$, $F$),
    kde: }
    \begin{itemize}
        \item[$\bullet$] \emph{$Q$ je konečná množina} vnitřních (řídicích) stavů,
        \item[$\bullet$] \emph{$\Sigma$ je konečná} vstupní abeceda,
        \item[$\bullet$] \emph{$\Gamma$ je konečná} zásobníková abeceda,
        \item[$\bullet$] \emph{$\delta$ je} přechodová funkce $Q \times (\Sigma \cup \{\epsilon\}) \times
        \Gamma^*\rightarrow$ $ 2^{Q \times \Gamma^*}$,
        \item[$\bullet$] \emph{$q_0 \in Q$ je} počáteční stav, $Z_0 \in \Gamma$ \emph{je} startovací symbol zásobníku \emph{a $F \subseteq Q$ je množina} koncových stavů.
        
    \end{itemize}
    
    Nechť $P = (Q,\Sigma,\Gamma,\delta,q_0,Z_0,F)$ je rozšířený zásob\-níkový automat. Konfigurací nazveme trojici $(q, \omega, \alpha) \in Q \times \Sigma^* \times \Gamma^*$, kde $q$ je aktuální stav vnitřního řízení, $\omega$ je dosud nezpracovaná část vstupního řetězce a $\alpha = Z_{i_1} Z_{i_2}$~\dots $Z_{i_k}$ je obsah zásobníku\footnote{$Z_{i_1}$ je vrchol zásobníku}.
    
\end{definition}
\subsection{Podsekce obsahující větu a odkaz}

\begin{definition}\label{Definice2}
Řetězec $\omega$ nad abecedou $\Sigma$ je přijat RZA $A$ \emph{jestliže $(q_0, \omega, Z_0) \overset{*}{\underset{A}{\vdash}} (q_F,\epsilon,\gamma)$ pro nějaké $\gamma \in \Gamma^*$ a $q_F~\in~F$. Množinu $L(A)  = \{ \omega~|~\omega$ je přijat RZA $A$\} $\subseteq$ $\Sigma^*$ nazýváme} jazyk přijímaný RZA $A$.

\end{definition}
Nyní si vyzkoušíme sazbu vět a důkazů opět s použitím balíku \verb|amsthm|. 
\begin{veta}
\emph{Třída jazyků, které jsou přijímány ZA, odpovídá} bezkontextovým jazykům.
\end{veta}

\begin{proof}
    V důkaze vyjdeme z Definice \ref{Definice} a \ref{Definice2}.
\end{proof}

\section{Rovnice a odkazy}
Složitější matematické formulace sázíme mimo plynulý text. Lze umístit několik výrazů na jeden řádek, ale pak je třeba tyto vhodně oddělit, například příkazem \verb|\quad|.
$$
\sqrt[i]{x_i ^3} \quad
\text{kde } x_i  \text{ je } i \text{-té sudé číslo splňující} \quad
x_i ^{x_i ^{i^2} +2} \leq y_i ^{x_i ^4}
$$

V rovnici (\ref{rov1}) jsou využity tři typy závorek s různou explicitně definovanou velikostí.

 \begin{eqnarray}
 \label{rov1}
    x &=& \bigg[\Big\{\big[a + b\big] * c\Big\}^d \oplus 2 \bigg]^{3/2} \\
    y &=& \lim_{x\to\infty}\frac{\frac{1}{\log_{10}x}}{\sin^2 x + \cos^2 x} \nonumber
 \end{eqnarray}
 
V~této větě vidíme, jak vypadá implicitní vysázení limity $ \lim_{n\to\infty} f(n) $
v~normálním odstavci textu. Podobně je\,to\,i~s~dalšími symboly jako $\prod_{i=1}^n2^i$ či $\bigcap_{A \in\mathcal{B}} A$. V pří\-padě vzorců $\lim\limits_{n\to\infty} f(n) $ a $
\prod\limits_{i=1}^n 2^i$ jsme si vynutili méně úspornou sazbu příkazem \verb|\limits|.

\begin{eqnarray}
   \int^a_b g(x)\,\mathrm{d}x &=& -\int\limits_a ^b f(x)\,\mathrm{d}x
\end{eqnarray}
\section{Matice}
Pro sázení matic se velmi často používá prostředí \verb|array| a závorky (\verb|\left|,\verb|\right|).
$$
\left(
\begin{array}{ccc}
a - b & \widehat{\xi + \omega} & \pi\\
\vec {\mathbf{a}} & \overleftrightarrow{AC} & \hat{\beta}
\end{array}
\right)
=1 \Longleftrightarrow \mathcal{Q} = \mathbb{R}
$$
$$
\mathbf{A} =
\left\|
\begin{array}{cccc}
a_{11} & a_{12} & \ldots & a_{1n} \\
a_{21} & a_{22} & \ldots & a_{2n} \\
\vdots & \vdots & \ddots & \vdots \\
a_{m1} & a_{m2} & \ldots & a_{mn} \\
\end{array}
\right\|
=
\left|
\begin{array}{cc}
 t & u \\ v & w
\end{array}
\right|
= tw - uv
$$

Prostředí \verb|array| lze úspěšně využít i jinde.
$$
\binom{n}{k} =
\left\{
\begin{array}{cl}
0 & \text{pro } k < 0 \text{ nebo } k > n \\
\frac{n!}{k!(n-k)!} & \text{pro } 0 \leq k \leq n\text{.}
\end{array}
\right.
$$


\end{document}

